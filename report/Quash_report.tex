% Created 2021-10-21 Thu 23:45
% Intended LaTeX compiler: xelatex
\documentclass[12pt]{article}
\usepackage{graphicx}
\usepackage{grffile}
\usepackage{longtable}
\usepackage{wrapfig}
\usepackage{rotating}
\usepackage[normalem]{ulem}
\usepackage{amsmath}
\usepackage{textcomp}
\usepackage{amssymb}
\usepackage{capt-of}
\usepackage{hyperref}
\usepackage{minted}
\usepackage{amsmath}
\usepackage{amssymb}
\usepackage{setspace}
\usepackage{subcaption}
\usepackage{mathtools}
\usepackage{xfrac}
\usepackage[margin=1in]{geometry}
\usepackage{marginnote}
\usepackage[utf8]{inputenc}
\usepackage{color}
\usepackage{epsf}
\usepackage{tikz}
\usepackage{graphicx}
\usepackage{pslatex}
\usepackage{hyperref}

%\usepackage{beton}
%\usepackage{euler}
%\usepackage[OT1]{fontenc}

\usepackage[T1]{fontenc}
\usepackage{newpxtext,eulerpx}

\usepackage{textgreek}
\renewcommand*{\textgreekfontmap}{%
{phv/*/*}{LGR/neohellenic/*/*}%
{*/b/n}{LGR/artemisia/b/n}%
{*/bx/n}{LGR/artemisia/bx/n}%
{*/*/n}{LGR/artemisia/m/n}%
{*/b/it}{LGR/artemisia/b/it}%
{*/bx/it}{LGR/artemisia/bx/it}%
{*/*/it}{LGR/artemisia/m/it}%
{*/b/sl}{LGR/artemisia/b/sl}%
{*/bx/sl}{LGR/artemisia/bx/sl}%
{*/*/sl}{LGR/artemisia/m/sl}%
{*/*/sc}{LGR/artemisia/m/sc}%
{*/*/sco}{LGR/artemisia/m/sco}%
}
\makeatletter
\newcommand*{\rom}[1]{\expandafter\@slowromancap\romannumeral #1@}
\makeatother
\DeclarePairedDelimiterX{\infdivx}[2]{(}{)}{%
#1\;\delimsize\|\;#2%
}
\newcommand{\infdiv}{D\infdivx}
\DeclarePairedDelimiter{\norm}{\left\lVert}{\right\rVert}
\DeclarePairedDelimiter{\ceil}{\left\lceil}{\right\rceil}
\DeclarePairedDelimiter{\floor}{\left\lfloor}{\right\rfloor}
\def\Z{\mathbb Z}
\def\R{\mathbb R}
\def\C{\mathbb C}
\def\N{\mathbb N}
\def\Q{\mathbb Q}
\def\noi{\noindent}
\onehalfspace
\usemintedstyle{bw}
\author{Sandy Urazayev, Jacob McNamee\thanks{University of Kansas (ctu@ku.edu)}}
\date{286; 12021 H.E.}
\title{Quash Report\\\medskip
\large EECS 678 Project \rom{1}}
\hypersetup{
 pdfauthor={Sandy Urazayev, Jacob McNamee},
 pdftitle={Quash Report},
 pdfkeywords={},
 pdfsubject={},
 pdfcreator={Emacs 28.0.50 (Org mode 9.4.6)}, 
 pdflang={English}}
\begin{document}

\maketitle
\tableofcontents


\section{Building and running}
\label{sec:org419df39}
You have to have a proper go 1.17 installation to build it. Simply run
\begin{minted}[frame=lines,fontsize=\footnotesize,obeytabs,mathescape,numbersep=5pt,numbersep=2mm,xleftmargin=0.25in]{sh}
go get -u -v ./...
\end{minted}
to get all the dependencies and then run
\begin{minted}[frame=lines,fontsize=\footnotesize,obeytabs,mathescape,numbersep=5pt,numbersep=2mm,xleftmargin=0.25in]{sh}
go build -v
\end{minted}
to build it. You will get an executable \texttt{quash} in your directory. Simply run
it. We also provide a makefile, running \texttt{make} will do it all for you.

To execute quash, type in \texttt{./quash} or run
\begin{minted}[frame=lines,fontsize=\footnotesize,obeytabs,mathescape,numbersep=5pt,numbersep=2mm,xleftmargin=0.25in]{sh}
go install ./...
\end{minted}
to run it globally by simply invoking \texttt{quash} anywhere on your system. The
binary is fully independent.
\section{Quash}
\label{sec:org930372a}
Quash is the best shell in the entirety of our existence. Let's walk through
how it's built!
\section{Forking and Executing}
\label{sec:orgf6a981b}
Quash is implemented in \href{https://golang.org}{Go}, which itself is a garbage-collected language that
runs threads to maintain the language runtime. When we want to fork from a Go
application, the forking will only spawn a copy of the thread that initiated
the forking. Therefore this new subprocess that just got forked lacks all the
supporting threads that Go applications \textbf{absolutely must have} for adequate
runtime performance. Therefore, Go does allow us to call \texttt{fork}, but we \textbf{have} to
run \texttt{exec} immediately, such that the call and execution stack is immediately
replaced by the newly loaded program.

This is achieved by \texttt{syscall.ForkExec} library call that only spawns a
subprocess with a loaded program and returns the new process's \texttt{pid}.
\begin{minted}[frame=lines,fontsize=\footnotesize,obeytabs,mathescape,numbersep=5pt,numbersep=2mm,xleftmargin=0.25in]{go}
pid, err := syscall.ForkExec(
        paths, args, &syscall.ProcAttr{
                Dir:   string,
                Env:   []string,
                Files: []uintptr,
                Sys:   &syscall.SysProcAttr{},
        })
\end{minted}
Notice that we have to pass in a couple of parameters, where \texttt{Dir} is the
current active directory where we are located, \texttt{Env} is a slice of strings, which
contains our environmental variables, \texttt{Files} is a slice of unsigned file
descriptor pointer values, and \texttt{Sys} is a struct to pass additional options.
\section{\texttt{PATH}}
\label{sec:org00b956b}
In order to run executables, we have to have a list of directors where we
would look for one. For this, we have our \texttt{PATH} environmental variable. Quash
solves this problem rather simply by going through all the directories in \texttt{PATH}
and searching for an exact executable name match in their globs. The
binary finding code is below
\begin{minted}[frame=lines,fontsize=\footnotesize,obeytabs,mathescape,numbersep=5pt,numbersep=2mm,xleftmargin=0.25in]{go}
// lookPath tries to find an absolute path to an executable
// name by searching directories on the PATH
// If the name is an absolute path or a shortened path (./)
// then this path is returned
func lookPath(name string) (string, error) {
        if filepath.IsAbs(name) { //if the user has absolute path then we good
                return name, nil
        }

        absPath := filepath.Join(currDir, name)
        _, err := os.Stat(absPath)
        if !os.IsNotExist(err) {
                return absPath, nil
        }
        path := getenv("PATH")
        if path == "" {
                err := errors.New("executable not found")
                return "", err
        }
        directories := strings.Split(path, ":")
        for _, directory := range directories {
                dirInfo, err := os.ReadDir(directory)
                if err != nil {
                        //quashError("%s : %s", errors.Unwrap(err), directory)
                        continue
                }
                for _, file := range dirInfo {
                        if file.Name() == name && !file.IsDir() {
                                return directory + "/" + name, nil
                        }
                }
        }
        err = errors.New("executable not found")
        return "", err

}
\end{minted}
Notice that the function would return the full path for a binary (example if
\texttt{PATH = /usr/bin} and executable is \texttt{echo}, \texttt{lookPath} would return
\texttt{/usr/bin/echo}). \texttt{getenv} and \texttt{setenv} are our user-defined functions that access
the global variable \texttt{myEnv}, which holds all of our active environmental variables.
\section{Pipes}
\label{sec:orgedec963}
Quash allows the user to sequentially run multiple programs while passing the
output data from one program to the input data of the next program in the
sequence. This is accomplished with the use of pipes. When Quash receives a
command, it separates the command into the programs the command wants us to
run and creates pipes to connect the processes to be created.
\begin{minted}[frame=lines,fontsize=\footnotesize,obeytabs,mathescape,numbersep=5pt,numbersep=2mm,xleftmargin=0.25in]{go}
// split input into different commands to be executed
commands := strings.Split(input, "|")
for index, command := range commands {
        commands[index] = strings.TrimSpace(command)
        args := strings.Split(commands[index], " ")
        args[0] = strings.TrimSpace(args[0])
        if builtinFunc, ok := builtins[args[0]]; ok && len(commands) == 1 {
                builtinFunc(args)
                addToHistory(input)
                return
        } else if ok {
                quashError("built-in command inside pipe chain")
                return
        }
}

pipeRead, pipeWrite := createPipes(len(commands) - 1)

\end{minted}
While the processes are being created (see Forking and Executing), the
processes are assigned a custom file descriptor table created using the
\texttt{fileDescriptor()} function. If there are pipes present in the command, then
\texttt{fileDescriptor()} will use the created pipes as files in the descriptor table,
overwriting the default behavior that uses the operating system’s standard
input (\texttt{stdin}) and standard output (\texttt{stdout}).
\begin{minted}[frame=lines,fontsize=\footnotesize,obeytabs,mathescape,numbersep=5pt,numbersep=2mm,xleftmargin=0.25in]{go}
// fileDescriptor returns a custom file descriptor for a call to ForkExec
// if there is only one command with no pipes, Stdin Stdout and Stderr are used
// pipes overwrite read, write, or both for processes inside of a pipe chain.
func fileDescriptor(
        index int,
        readPipe []*os.File,
        writePipe []*os.File,
        in *os.File,
        out *os.File,
        err *os.File,
) []uintptr {
        // One command, so no pipes
        if len(readPipe) == 0 {
                return []uintptr{
                        in.Fd(),
                        out.Fd(),
                        err.Fd(),
                }
        }
        // first in a chain
        if index == 0 {
                return []uintptr{
                        in.Fd(),
                        writePipe[0].Fd(),
                        err.Fd(),
                }
        }
        // last in a chain
        if index == len(readPipe) { ... }
        // middle of a chain
        return []uintptr{ ... }
}
\end{minted}
Finally, we must close the pipes within the quash process in order to properly
transmit EOF when a child process finishes execution. This is done using the
\texttt{closePipe()} function, which closes the pipe ends that we distributed to the
child process using the \texttt{fileDescriptor()} function.
\begin{minted}[frame=lines,fontsize=\footnotesize,obeytabs,mathescape,numbersep=5pt,numbersep=2mm,xleftmargin=0.25in]{go}
// closePipe closes used pipe ends based on where they are in a chain of piped
// commands if only one command exists, there are no pipes and this function
// does nothing.
func closePipe(index int, readPipe []*os.File, writePipe []*os.File) {
        // One command, so no pipes
        if len(readPipe) == 0 {
        } else if index == 0 {
                // first in a chain
                writePipe[0].Close()
        } else if index == len(readPipe) {
                // last in a chain
                readPipe[index-1].Close()
        } else {
                // middle of a chain
                readPipe[index-1].Close()
                writePipe[index].Close()
        }
}
\end{minted}
Note that in C you would have to also close excess pipes between the fork and
execute function calls in the child process, but in Go we only assigned the
child process the necessary pipes, so no additional pipes need to be closed. 
\section{Background Processes}
\label{sec:org13fbdd0}
Like many other shell programs, Quash has the ability to execute programs in
either the foreground or the background. A program or group of programs
running in the background is called a job. A program is designated to run in
the background as a job by adding the \& character to the end of the command. A
set of programs linked by pipes can also be run in the background the same
way, using a single \& at the very end. For example, \texttt{ls \&} and \texttt{ls | wc \&} both
create jobs that will execute in the background.
\begin{minted}[frame=lines,fontsize=\footnotesize,obeytabs,mathescape,numbersep=5pt,numbersep=2mm,xleftmargin=0.25in]{go}
// job is the struct that holds info about background processes
type job struct {
        // pid associated with currently running process in the job
        pid int
        // jid associated with this job
        jid int
        // command that created this job
        command string
        // reference to the current process
        process *os.Process
}
\end{minted}
Each job in an instance of Quash will be assigned a unique job
identifier (\texttt{jid}). Jobs are referenced using these identifiers when using
built in commands such as jobs or kill (see \textbf{Builtins}). Additionally,
each job will print a message when they are first created and when they
terminate. If one process within a pipe chain terminates with an error,
the job will terminate. 
\section{Builtins}
\label{sec:org3598adc}
Quash has a handful of pre-defined keywords that perform special functionality
for the user. These commands are: \texttt{exit}, \texttt{quit}, \texttt{set}, \texttt{cd}, \texttt{kill}, \texttt{jobs}, and
\texttt{history}. These built in functions cannot be executed as part of a chain of
processes, as they are not themselves process. Instead they are functions that
manipulate aspects of the shell, such as changing the environment. 
\subsection{\texttt{quit} / \texttt{exit}}
\label{sec:orge3d3828}
\texttt{quit} and \texttt{exit} are aliases for the same function within Quash. This function
terminates Quash.
\begin{minted}[frame=lines,fontsize=\footnotesize,obeytabs,mathescape,numbersep=5pt,numbersep=2mm,xleftmargin=0.25in]{sh}
Usage: quit or exit
\end{minted}
\subsection{\texttt{set}}
\label{sec:orgccec4c4}
\texttt{set} allows the user to change environment variables, such as the current
working PATH. The initial variables and values are set by the OS. \texttt{set} can
also add a new variable to Quash’s environment (but not the OS’s
environment).
\begin{minted}[frame=lines,fontsize=\footnotesize,obeytabs,mathescape,numbersep=5pt,numbersep=2mm,xleftmargin=0.25in]{sh}
Usage: set variable
\end{minted}
where \texttt{variable} is the name of the variable to add or update, and value is the
value to set variable as.
\subsection{\texttt{cd}}
\label{sec:org034ebc4}
\texttt{cd} stands for change directory. cd changes the current directory that Quash
is working within.
\begin{minted}[frame=lines,fontsize=\footnotesize,obeytabs,mathescape,numbersep=5pt,numbersep=2mm,xleftmargin=0.25in]{sh}
Usage: cd directory
\end{minted}
where \texttt{directory} is an absolute or relative path to change to. If no
directory is specified, then cd will change the directory to the \texttt{\$HOME}
directory specified in Quash’s environment.
\subsection{\texttt{kill}}
\label{sec:orgbfa223e}
\texttt{kill} allows the user to manually send signals to a currently executing
job. This is especially useful for sending signals to forcefully end the job,
hence the name \texttt{kill}.
\begin{minted}[frame=lines,fontsize=\footnotesize,obeytabs,mathescape,numbersep=5pt,numbersep=2mm,xleftmargin=0.25in]{sh}
Usage: kill signal jid
\end{minted}
where \texttt{signal} is the number of the signal you wish to send (check your OS to
see what number each signal corresponds to) and \texttt{jid} is the job identification
number corresponding to the job you wish to signal.
\subsection{\texttt{jobs}}
\label{sec:orgedc422f}
\texttt{jobs} prints all currently executing background jobs.
\begin{minted}[frame=lines,fontsize=\footnotesize,obeytabs,mathescape,numbersep=5pt,numbersep=2mm,xleftmargin=0.25in]{sh}
Usage: jobs
\end{minted}
Output: \texttt{[jid] pid running in background} where \texttt{jid} is the job identification
number for the job and \texttt{pid} is the process identification number for the
currently executing process within the job. This line is printed for each
currently running job, sorted by \texttt{jid}.
\subsection{\texttt{history}}
\label{sec:org1fdf7a4}
\texttt{history} prints a list of all previous valid commands used within the current
execution of Quash. If the command failed, such as misspelling an executable
name, the command will not be added to the history.
\begin{minted}[frame=lines,fontsize=\footnotesize,obeytabs,mathescape,numbersep=5pt,numbersep=2mm,xleftmargin=0.25in]{sh}
Usage: history
\end{minted}
Output: \texttt{number cmd} where \texttt{number} is the index of the command starting
at 1 and \texttt{cmd} is the entire text of the previous command. This line is printed
for every previous valid command, sorted by number. 
\section{Arrow Keys}
\label{sec:org2b245c8}
We support arrow key movements! We do this by manually catching keyboard
interrupts from \texttt{/dev/tty} with \texttt{keyboard} interface and then depending on each
key pressed, we decide on what to do. This actually changes the input
logistics completely, as in when the user presses a key, it doesn't get
flushed onto the screen, we swallow it and must decide what to do with it. We
catch all the special keys and then print all printable characters we
caught. The subroutine for it looks like the following
\begin{minted}[frame=lines,fontsize=\footnotesize,obeytabs,mathescape,numbersep=5pt,numbersep=2mm,xleftmargin=0.25in]{go}
// takeInput reads a newline-terminated input from a bufio reader
func takeInput(reader *bufio.Reader) string {
        if err := keyboard.Open(); err != nil {
                panic(err)
        }
        defer func() {
                _ = keyboard.Close()
        }()

        cmdNum := len(goodHistory)
        var readCharacter rune
        input := ""
        curPosition := 0

        for {
                char, key, err := keyboard.GetKey()
                if err != nil {
                        quashError("bad input: %s", err.Error())
                }
                readCharacter = char

                // See what key we actually pressed, I tried doing switch
                // but it works kinda wonky. If statements forever <3
                // --------------------------------------------------

                // On enter, flush a newline and return whatever we have
                if key == keyboard.KeyEnter {
                        fmt.Fprint(os.Stdout, NEWLINE)
                        return input + string(char)
                }
                // On Ctrl-D or Escape just close the shell altogether
                if key == keyboard.KeyEsc {
                        if isTerminal {
                                fmt.Fprint(os.Stdout, NEWLINE)
                        }
                        exit(nil)
                }
                // Only exit on Ctrl-D if input is empty
                if key == keyboard.KeyCtrlD {
                        if curPosition != 0 || len(input) != 0 {
                                continue
                        }
                        if isTerminal {
                                fmt.Fprint(os.Stdout, NEWLINE)
                        }
                        exit(nil)
                }
                // On a space just set readCharacter to a space run
                if key == keyboard.KeySpace {
                        readCharacter = ' '
                }
                // On backspace, move cursor to the left, clean character,
                // and move the cursor again to the left. Delete last input element
                if key == keyboard.KeyBackspace || key == keyboard.KeyBackspace2 {
                        // If cursor is already at the home position, don't move
                        if curPosition < 1 {
                                continue
                        }
                        fmt.Fprintf(os.Stdout, "\b \b")
                        input = input[:curPosition-1]
                        curPosition--
                        continue
                }
                // On arrow up press, clean out the terminal and replace the user input
                // with whatever previous good command we can find. Works on multiple
                // arrow up key presses too
                if key == keyboard.KeyArrowUp {
                        if len(goodHistory) < 1 {
                                continue
                        }
                        // Clear the input first
                        resetTermInput(len(input))
                        cmdNum = prevCmdNum(cmdNum)
                        input = printOldGoodCommand(cmdNum)
                        curPosition = len(input)
                        continue
                }
                // On arrow down press, clean out the terminal and replace with whatever
                // command came after. Only makes sense if run after one or mory presses
                // of the arrow up key. On the bottom it will set user input to just clean
                if key == keyboard.KeyArrowDown {
                        if len(goodHistory) < 1 {
                                continue
                        }
                        resetTermInput(len(input))
                        // If at the end of history, just clear the input
                        if cmdNum >= len(goodHistory)-1 {
                                input = ""
                                cmdNum = len(goodHistory)
                                curPosition = 0
                                continue
                        }
                        // Get the later good command
                        cmdNum = nextCmdNum(cmdNum)
                        input = printOldGoodCommand(cmdNum)
                        curPosition = len(input)
                        continue
                }
                // Ignore left and right arrow keys
                if key == keyboard.KeyArrowLeft || key == keyboard.KeyArrowRight {
                        continue
                }
                // Send kill signals if ctrl is encountered or clear the input
                if key == keyboard.KeyCtrlC {
                        // Don't do anything if we have an empty command
                        if curPosition == 0 && len(input) == 0 {
                                sigintChan <- syscall.SIGINT
                                continue
                        }
                        fmt.Fprintf(os.Stdout, "\033[41m^C\033[0m\n")
                        input = ""
                        curPosition = 0
                        greet()
                        continue
                }
                // Ctrl-L should clear the screen
                if key == keyboard.KeyCtrlL {
                        executeInput("clear")
                        greet()
                        // Reprint whatever we had before
                        fmt.Fprintf(os.Stdout, "%s", input)
                        continue
                }
                // If the character is NOT printable, skip saving it
                if !unicode.IsPrint(readCharacter) {
                        continue
                }
                // Print the character that we swallowed up and append to input
                fmt.Fprint(os.Stdout, string(readCharacter))
                input += string(readCharacter)
                curPosition = len(input)
        }
}l
\end{minted}
\end{document}